\subsection{Responsibility-Driven Analysis }

\anoun{Travpedia} is an online travel and accommodation \anoun{booking system}.

\emph{Travel and accommodation companies are able to subscribe to Travpedia for
a monthly subscription cost of £200 plus an initial £50 joining fee. This
subscription allows the company to offer their products on the Travpedia
website where they can be purchased by visiting users.}

\anoun{Products} that are \averb{advertised} on the Travpedia \anoun{website}
include \anoun{accommodation}, \anoun{package holidays} and \anoun{travel} by
\anoun{air}, \anoun{rail} and \anoun{sea,}

Visitors to the website, after \averb{registering} an \anoun{account}, are able
to \averb{search} for all available products \averb{offered} by these
subscribed \anoun{companies}. They are able to search with a number of
\anoun{criteria} including \anoun{type of product}, \anoun{location},
\anoun{date} and \anoun{price}. Users can then view these \anoun{search
results} and \averb{book} and \averb{pay} for products through the website.
Users may also \averb{rate} and \averb{review} individual products and services
that they have \averb{purchased}. A product gains a review score based on
these ratings. This review rating system provides further search criteria
whereby a user can \averb{filter} search results by \anoun{review score}.

\anoun{Payments} made by both subscribing companies and users are
\averb{handled} online by a third party \anoun{consortium}.
\anoun{Subscribers} must pay by \anoun{debit} or \anoun{credit card} while
users have the additional option of paying with gift \anoun{vouchers} offered
by Travpedia.

Users are able to \averb{view }\anoun{bookings} they have made and, where
possible, \averb{cancel} these bookings and \averb{receive} a \anoun{refund}
via the third party consortium.

\emph{Travpedia disseminates advertisements and promotional offers to users
based on previous patterns of use and previous purchases.  These personalised
offering are sent to mobile phones through SMS and email accounts.}

Users may \averb{opt out} of receiving phone and email \anoun{alerts}.
