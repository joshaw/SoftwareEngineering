\subsection{Responsibility-Driven Analysis }

\anoun{Travpedia} is an online travel and accommodation \anoun{booking system}.

\emph{Travel and accommodation companies are able to subscribe to Travpedia for
	a monthly subscription cost of £200 plus an initial £50 joining fee. This
	subscription allows the company to offer their products on the Travpedia
website where they can be purchased by visiting users.}

\anoun{Products} that are \averb{advertised} on the Travpedia \anoun{website}
include \anoun{accommodation}, \anoun{package holidays} and \anoun{travel} by
\anoun{air}, \anoun{rail} and \anoun{sea,}

Visitors to the website, after \averb{registering} an \anoun{account}, are able
to \averb{search} for all available products \averb{offered} by these
subscribed \anoun{companies}. They are able to search with a number of
\anoun{criteria} including \anoun{type of product}, \anoun{location},
\anoun{date} and \anoun{price}. Users can then view these \anoun{search
results} and \averb{book} and \averb{pay} for products through the website.
Users may also \averb{rate} and \averb{review} individual products and services
that they have \averb{purchased}. A product gains a review score based on
these ratings. This review rating system provides further search criteria
whereby a user can \averb{filter} search results by \anoun{review score}.

\anoun{Payments} made by both subscribing companies and users are
\averb{handled} online by a third party \anoun{consortium}.
\anoun{Subscribers} must pay by \anoun{debit} or \anoun{credit card} while
users have the additional option of paying with gift \anoun{vouchers} offered
by Travpedia.

Users are able to \averb{view }\anoun{bookings} they have made and, where
possible, \averb{cancel} these bookings and \averb{receive} a \anoun{refund}
via the third party consortium.

\emph{Travpedia disseminates advertisements and promotional offers to users
	based on previous patterns of use and previous purchases.  These personalised
offering are sent to mobile phones through SMS and email accounts.}

Users may \averb{opt out} of receiving phone and email \anoun{alerts}.

\subsection{Nouns}

\renewcommand*{\arraystretch}{1.3}
% \begin{longtabu}{m{2.8cm} >{\centering\arraybackslash}m{1.6cm} >{\centering\arraybackslash}m{2.8cm} X[m]}
\begin{longtable}{m{2.8cm} >{\centering\arraybackslash}m{1.6cm} >{\centering\arraybackslash}m{2.8cm} m{5.8cm}}
	\toprule
	\textbf{Noun (Cont.)} & \textbf{Accepted} & \textbf{Class Name} & \textbf{Rationale} \\
	\endhead
	\toprule
	\textbf{Noun} & \textbf{Accepted as Class} & \textbf{Class Name} & \textbf{Rationale} \\
	\endfirsthead
	\midrule
	Travpedia \newline
	booking system \newline
	website
	& No &  & The whole system is the Travpedia website \\

	\midrule
	account \newline
	user \newline
	visitor
	& Yes & User & Class will hold user information, payment details,
	mailing preferences and details of bookings \\

	\midrule
	advertisement \newline
	promotional offer
	& No &  & Part of subscriber promotions subsystem, outside scope \\

	\midrule
	product \newline
	service
	& Yes & Product & Superclass \\

	\midrule
	type of product
	& No &  & This is defined by the subclasses of Product \\

	\midrule
	accommodation
	& Yes & Accommodation & Subclass of Product \\

	\midrule
	package
	& Yes & PackageHoliday & Subclass of Product \\

	\midrule
	travel
	& Yes & Travel & Subclass of Product \\

	\midrule
	air
	& Yes & Flight & Subclass of Travel, subclasses used because different
	forms of travel have some common attributes and some distinct ones \\

	\midrule
	rail
	& Yes & Rail & Subclass of Travel \\

	\midrule
	sea
	& Yes & Ferry & Subclass of Travel \\

	\midrule
	alerts
	& No &  & Part of promotional mailing subsystem \\

	\midrule
	booking
	& Yes & Booking & Class holds all information about an individual booking \\

	\midrule
	company \newline
	subscriber
	& No &  & Part of subsystem that handles company subscriptions \\

	\midrule
	consortium
	& No &  & Part of payment sub-system \\

	\midrule
	payment \newline
	credit card \newline
	debit card \newline
	gift voucher \newline
	refund
	& Yes & Payment & Class holds information relating to a payment (refund is a negative payment) \\

	\midrule
	search results
	& Yes & SearchResults & Class holds result of an individual search \\

	\midrule
	criteria \newline
	date \newline
	price \newline
	location
	& No &  & These are attributes of SearchResults \\

	\midrule
	review
	& Yes & Review & Class holds a product review \\

	\midrule
	review score \newline
	product rating & No &  & These are attributes of a review \\
	\bottomrule
\end{longtable}

\subsection{Verbs}

\renewcommand*{\arraystretch}{1.3}
\begin{longtable}{m{2.5cm} >{\centering\arraybackslash}m{2cm} m{6cm}}
	\toprule
	\textbf{Verb} & \textbf{Accepted} & \textbf{Rationale} \\
	\midrule
	\endhead
	\toprule
	\textbf{Verb} & \textbf{Accepted as method} & \textbf{Rationale (if rejected)} \\
	\endfirsthead
	\midrule
	advertise & No & Out of Scope \\
	\midrule
	register & Yes &  \\
	\midrule
	search & Yes &  \\
	\midrule
	offer &  & Generic term \\
	\midrule
	book & Yes &  \\
	\midrule
	pay & Yes &  \\
	\midrule
	rate & No & Out of Scope \\
	\midrule
	review & No & Out of Scope \\
	\midrule
	purchase & No & Is encompassed by book and pay \\
	\midrule
	filter & Yes &  \\
	\midrule
	handle & No & Generic term \\
	\midrule
	view & Yes &  \\
	\midrule
	cancel & Yes &  \\
	\midrule
	receive &  & Out of Scope \\
	\midrule
	opt out & No & This would be accomplished by setting a field variable in
	the User, getters and setters are assumed and are not explicitly
	included here \\
	\bottomrule
\end{longtable}
