\section{Specification}

Travpedia is an online travel and accommodation booking system. Travel and
accommodation companies are able to subscribe to Travpedia for a monthly
subscription cost of £200 plus an initial £50 joining fee.  This subscription
allows the company to offer their products on the Travpedia website where they
can be purchased by visiting users. Products that are advertised on the
Travpedia website include accommodation, package holidays and travel by air,
rail and sea. One or more products may be combined into a single booking.

Visitors to the website, after registering an account, are able to search for
all available products offered by these subscribed companies.  They are able to
search with a number of criteria including type of product, location, date and
price. Users can then view these search results and book and pay for products
through the website. Users may also rate and review accommodation and package
holidays that they have purchased. A product gains a review score based on
these ratings.  This review rating system provides further search criteria
whereby a user can filter search results by review score.

Payments made by both subscribing companies and users are handled online by a
third party consortium. Subscribers must pay by debit or credit card while
users have the additional option of paying with gift vouchers offered by
Travpedia.

Users are able to view bookings they have made and, where possible, cancel
these bookings and receive a refund via the third party consortium.

Travpedia disseminates advertisements and promotional offers to users based on
previous patterns of use and previous purchases. These personalised offering
are sent to mobile phones through SMS and email accounts.  Users may opt out of
receiving SMS and email alerts.

Travpedia also has a number of critical compliance and security requirements.
Travpedia stores users' personal information and payment details that should
not be disclosed to other parties or kept for any longer than necessary. If
this data is maliciously accessed, disclosed, leaked or manipulated it could
breach confidentiality and data protection.  Furthermore, any transaction
information sent to the third party consortium used for payment must be kept
secure. This is done with encryption using 128-bit SSL certificates.

During peak time, Travpedia receives up to one million simultaneous users and
is designed to handle this number of users. The design is also scalable to
accommodate its growing number of users and subscribers.  This system is used
by users 24 hours a days and must be always available.  All data that Travpedia
stores must also be kept safe from system failures. For this reason, user
account details, itinerary and transaction data and subscribers' data are
stored and backed up on three database servers in three distinct locations. Two
of these locations are in the UK and the other is in the USA. This allows
Travpedia to minimise downtime after unforeseen system failures.

\subsection{Scope}

The scope of our design is limited primarily to subsystems devoted to user's
interactions with the Travpedia website. In particular, we will consider the
following:

\begin{itemize}
	\item Travpedia account registration.
	\item Searching for products.
	\item Making new bookings.
	\item Viewing and cancelling previous bookings.
\end{itemize}

\subsection{Assumptions}

We have made a number of assumptions regarding the external systems and
remaining internal modules described in the whole system specification.  We
assume that these other modules have their own interface with which we can
communicate when necessary.

\begin{description}
	\item [{Subscribers:}] All aspects of company subscriptions to Travpedia
		are handled by a separate module.
	\item [{Consortium:}] The third party consortium will deal with all aspects
		of forwarding user payments to Travpedia's subscribers. We will be
		responsible for sending user payments to this consortium who will then
		respond accordingly about the success of the transaction.
	\item [{User~Database:}] There is a database for user profile and account
		data.  We can make requests to this database to retrieve and update
		this data.
	\item [{Product~Database:}] There is a database for all travel and
		accommodation product data. We can make requests to this database to
		retrieve and update this data.
	\item [{Backup:}] In the case of both databases, we assume that another
		module handles ensuring this data is backed up across the three
		distinct database locations. Finally, this module also ensures that
		data is not held longer than necessary in order to comply with the Data
		Protection Act.
	\item [{Notifications:}] Users can request email and phone notifications.
		Our system will provide the user with the option to receive these, but
		a separate module deals with the selection and dissemination of this
		information.
\end{description}
